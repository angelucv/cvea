% Options for packages loaded elsewhere
% Options for packages loaded elsewhere
\PassOptionsToPackage{unicode}{hyperref}
\PassOptionsToPackage{hyphens}{url}
\PassOptionsToPackage{dvipsnames,svgnames,x11names}{xcolor}
%
\documentclass[
  11pt,
  a4paper,
]{article}
\usepackage{xcolor}
\usepackage{amsmath,amssymb}
\setcounter{secnumdepth}{-\maxdimen} % remove section numbering
\usepackage{iftex}
\ifPDFTeX
  \usepackage[T1]{fontenc}
  \usepackage[utf8]{inputenc}
  \usepackage{textcomp} % provide euro and other symbols
\else % if luatex or xetex
  \usepackage{unicode-math} % this also loads fontspec
  \defaultfontfeatures{Scale=MatchLowercase}
  \defaultfontfeatures[\rmfamily]{Ligatures=TeX,Scale=1}
\fi
\usepackage{lmodern}
\ifPDFTeX\else
  % xetex/luatex font selection
\fi
% Use upquote if available, for straight quotes in verbatim environments
\IfFileExists{upquote.sty}{\usepackage{upquote}}{}
\IfFileExists{microtype.sty}{% use microtype if available
  \usepackage[]{microtype}
  \UseMicrotypeSet[protrusion]{basicmath} % disable protrusion for tt fonts
}{}
\usepackage{setspace}
\makeatletter
\@ifundefined{KOMAClassName}{% if non-KOMA class
  \IfFileExists{parskip.sty}{%
    \usepackage{parskip}
  }{% else
    \setlength{\parindent}{0pt}
    \setlength{\parskip}{6pt plus 2pt minus 1pt}}
}{% if KOMA class
  \KOMAoptions{parskip=half}}
\makeatother
% Make \paragraph and \subparagraph free-standing
\makeatletter
\ifx\paragraph\undefined\else
  \let\oldparagraph\paragraph
  \renewcommand{\paragraph}{
    \@ifstar
      \xxxParagraphStar
      \xxxParagraphNoStar
  }
  \newcommand{\xxxParagraphStar}[1]{\oldparagraph*{#1}\mbox{}}
  \newcommand{\xxxParagraphNoStar}[1]{\oldparagraph{#1}\mbox{}}
\fi
\ifx\subparagraph\undefined\else
  \let\oldsubparagraph\subparagraph
  \renewcommand{\subparagraph}{
    \@ifstar
      \xxxSubParagraphStar
      \xxxSubParagraphNoStar
  }
  \newcommand{\xxxSubParagraphStar}[1]{\oldsubparagraph*{#1}\mbox{}}
  \newcommand{\xxxSubParagraphNoStar}[1]{\oldsubparagraph{#1}\mbox{}}
\fi
\makeatother


\usepackage{longtable,booktabs,array}
\usepackage{calc} % for calculating minipage widths
% Correct order of tables after \paragraph or \subparagraph
\usepackage{etoolbox}
\makeatletter
\patchcmd\longtable{\par}{\if@noskipsec\mbox{}\fi\par}{}{}
\makeatother
% Allow footnotes in longtable head/foot
\IfFileExists{footnotehyper.sty}{\usepackage{footnotehyper}}{\usepackage{footnote}}
\makesavenoteenv{longtable}
\usepackage{graphicx}
\makeatletter
\newsavebox\pandoc@box
\newcommand*\pandocbounded[1]{% scales image to fit in text height/width
  \sbox\pandoc@box{#1}%
  \Gscale@div\@tempa{\textheight}{\dimexpr\ht\pandoc@box+\dp\pandoc@box\relax}%
  \Gscale@div\@tempb{\linewidth}{\wd\pandoc@box}%
  \ifdim\@tempb\p@<\@tempa\p@\let\@tempa\@tempb\fi% select the smaller of both
  \ifdim\@tempa\p@<\p@\scalebox{\@tempa}{\usebox\pandoc@box}%
  \else\usebox{\pandoc@box}%
  \fi%
}
% Set default figure placement to htbp
\def\fps@figure{htbp}
\makeatother





\setlength{\emergencystretch}{3em} % prevent overfull lines

\providecommand{\tightlist}{%
  \setlength{\itemsep}{0pt}\setlength{\parskip}{0pt}}



 


\usepackage{fancyhdr}
\usepackage{titling}
\pagestyle{fancy}
\fancyhf{}
\fancyhead[L]{\includegraphics[height=0.5cm]{cvea-logo.png}}
\fancyhead[R]{\leftmark}
\fancyfoot[C]{\thepage}
\renewcommand{\headrulewidth}{0.4pt}
\pretitle{\begin{center}\includegraphics[width=1\textwidth]{cvea-logo.png}\\[\bigskipamount]}
\posttitle{\end{center}}
\makeatletter
\@ifpackageloaded{caption}{}{\usepackage{caption}}
\AtBeginDocument{%
\ifdefined\contentsname
  \renewcommand*\contentsname{Table of contents}
\else
  \newcommand\contentsname{Table of contents}
\fi
\ifdefined\listfigurename
  \renewcommand*\listfigurename{List of Figures}
\else
  \newcommand\listfigurename{List of Figures}
\fi
\ifdefined\listtablename
  \renewcommand*\listtablename{List of Tables}
\else
  \newcommand\listtablename{List of Tables}
\fi
\ifdefined\figurename
  \renewcommand*\figurename{Figure}
\else
  \newcommand\figurename{Figure}
\fi
\ifdefined\tablename
  \renewcommand*\tablename{Table}
\else
  \newcommand\tablename{Table}
\fi
}
\@ifpackageloaded{float}{}{\usepackage{float}}
\floatstyle{ruled}
\@ifundefined{c@chapter}{\newfloat{codelisting}{h}{lop}}{\newfloat{codelisting}{h}{lop}[chapter]}
\floatname{codelisting}{Listing}
\newcommand*\listoflistings{\listof{codelisting}{List of Listings}}
\makeatother
\makeatletter
\makeatother
\makeatletter
\@ifpackageloaded{caption}{}{\usepackage{caption}}
\@ifpackageloaded{subcaption}{}{\usepackage{subcaption}}
\makeatother
\usepackage{bookmark}
\IfFileExists{xurl.sty}{\usepackage{xurl}}{} % add URL line breaks if available
\urlstyle{same}
\hypersetup{
  pdftitle={Modelación y Proyección de la Mortalidad Neonatal, Postneonatal y en la niñez por entidad federal en Venezuela},
  pdfauthor={Lic. Arlet Moreno; Prof.~Angel Colmenares},
  pdfkeywords={Palabra Clave 1, Palabra Clave 2, Palabra Clave 3},
  colorlinks=true,
  linkcolor={blue},
  filecolor={Maroon},
  citecolor={Blue},
  urlcolor={Blue},
  pdfcreator={LaTeX via pandoc}}


\title{Modelación y Proyección de la Mortalidad Neonatal, Postneonatal y
en la niñez por entidad federal en Venezuela}
\author{Lic. Arlet Moreno \and Prof.~Angel Colmenares}
\date{2025-11-23}
\begin{document}
\maketitle
\begin{abstract}
\textbf{Resumen:} La planificación de riesgos demográficos y actuariales
en Venezuela ha dependido históricamente de proyecciones determinísticas
(tales como las utilizadas por CELADE e INDEC), que no logran capturar
la incertidumbre inherente ni la marcada heterogeneidad regional de la
mortalidad. Existe una preocupación crítica y la necesidad de una
estimación detallada del riesgo de fallecimiento en recién nacidos. Este
estudio aplicó el modelo estocástico de Lee-Carter (LC) a los datos
extraídos de los Anuarios de Mortalidad del Ministerio del Poder Popular
para la Salud (MPPS) para el período 1995-2012. La información fue
desagregada minuciosamente por Entidad Federal, sexo y edad con alta
resolución temporal (días y meses). El índice temporal \(\kappa_t\), que
captura la tendencia de la mortalidad, se modeló utilizando series de
tiempo y se proyectó hasta el año 2022, empleando el entorno R y sus
librerías especializadas como demography e ILC. Se identificó una severa
disparidad regional en el riesgo de mortalidad infantil. El estado Zulia
demostró consistentemente las tasas más elevadas en los períodos
Neonatal, Postneonatal y en la Niñez. El riesgo se concentra en el
período Neonatal (0-27 días), superando al riesgo Postneonatal. Un
hallazgo crítico fue la proyección de una anomalía en la tendencia para
el estado Trujillo, donde la tasa de mortalidad neonatal mostró un
ascenso proyectado a partir de 2014, lo que contrasta con la tendencia
general descendente del resto del país. La modelación estocástica a
nivel subnacional es indispensable para generar un mapa de riesgo
accionable. La alta concentración del riesgo en la fase Neonatal indica
debilidades sistémicas en la atención perinatal. Se concluye que la
planificación de la salud pública y la práctica actuarial deben
abandonar los supuestos de homogeneidad nacional y focalizar la
intervención con base en la estratificación regional del riesgo.
\end{abstract}


\setstretch{1.5}
\section{Introducción}\label{introducciuxf3n}

\subsection{Contexto y Relevancia (El Marco Actuarial
Venezolano)}\label{contexto-y-relevancia-el-marco-actuarial-venezolano}

La gestión de riesgos demográficos y financieros en Venezuela,
particularmente en lo concerniente a la previsión social y el sector
asegurador, ha enfrentado históricamente un desafío estructural derivado
de la obsolescencia de sus bases técnicas y la dependencia de métodos de
proyección simplistas. La ciencia actuarial tradicional, al igual que
las proyecciones de población y mortalidad utilizadas por entidades como
el Centro Latinoamericano y Caribeño de Demografía (CELADE) o el
Instituto Nacional de Estadística y Censos (INDEC), se ha apoyado en
modelos puramente determinísticos. Estos enfoques asumen que las
variables demográficas, aunque sujetas a cierto movimiento, son
relativamente suaves y moderadamente previsibles en el tiempo.

Esta dependencia del determinismo introduce una limitación crítica: se
ignora la incertidumbre inherente y la alta volatilidad propia de las
dinámicas sociales y sanitarias, especialmente en contextos de
desarrollo desigual y crisis. La probabilidad de que las estimaciones
futuras coincidan con el valor puntual proyectado se torna baja,
comprometiendo la sostenibilidad de los sistemas de pensiones, el
cálculo de primas de seguros de vida y la planificación sanitaria a
largo plazo.

El rol del actuario moderno se redefine como gestor de la incertidumbre.
Para subsanar estos inconvenientes, la aplicación de modelos
estocásticos resulta imprescindible. Estos modelos, a diferencia de los
determinísticos, permiten estimar una nube de valores probables y
proyectar los riesgos futuros con un determinado grado de confianza.
Este enfoque ofrece una visión más realista y gestionable del riesgo a
largo plazo, trascendiendo las simplificaciones de las proyecciones
puntuales y permitiendo la cuantificación de las primas de riesgo
necesarias para afrontar la variabilidad demográfica.

\subsection{Problema Específico: Modelación de la Volatilidad
Subnacional}\label{problema-especuxedfico-modelaciuxf3n-de-la-volatilidad-subnacional}

Históricamente, las proyecciones demográficas en Venezuela han dependido
de modelos puramente determinísticos, como los utilizados por CELADE e
INDEC, que asumen que las variables demográficas exhiben movimientos
relativamente suaves y moderadamente previsibles en el tiempo. Este
enfoque se ha vuelto insuficiente. La situación demográfica en un país
en vías de desarrollo y, más aún, su desagregación territorial (Entidad
Federal), muestra una alta volatilidad que no es homogénea.

El uso de proyecciones determinísticas introduce una limitación crítica:
la baja probabilidad de que las estimaciones futuras coincidan con el
valor proyectado, especialmente en entornos económicos y sociales
volátiles. Para subsanar estos inconvenientes, la aplicación de modelos
estocásticos resulta imprescindible. Estos modelos permiten estimar una
nube de valores futuros, conteniendo los valores proyectados con un
determinado grado de confianza, ofreciendo así una visión más realista y
gestionable del riesgo.

El foco de la presente investigación se centra en la alta preocupación
que genera la mortalidad de los recién nacidos en Venezuela, buscando
específicamente desagregar el riesgo en aquellos períodos (Neonatal,
Postneonatal) donde se suscitan la mayor cantidad de defunciones, y en
aquellas entidades geográficas donde se evidencian las mayores tasas.

\subsection{Justificación del Vacío (La Necesidad de la Desagregación
Estocástica)}\label{justificaciuxf3n-del-vacuxedo-la-necesidad-de-la-desagregaciuxf3n-estocuxe1stica}

El vacío de conocimiento abordado radica en la ausencia de proyecciones
de mortalidad estocásticas que ofrezcan una resolución lo
suficientemente alta, tanto a nivel geográfico como temporal, para ser
utilizadas en el diseño de políticas públicas focalizadas. La falta de
datos locales o la inaplicabilidad de modelos extranjeros en el entorno
económico venezolano actual es un problema bien documentado. Este
estudio llena ese vacío al aplicar un modelo de estándar internacional
(Lee-Carter) a datos locales, desagregando la información hasta el nivel
de Entidad Federal y edad en días/meses.

La capacidad de identificar qué no se sabe sobre el riesgo de mortalidad
infantil en Venezuela (es decir, dónde y cuándo ocurre la mortalidad de
manera más intensa) es la justificación principal. La desagregación
geográfica permite generar un mapa de riesgo accionable, indispensable
para que el sector salud pública y los entes de previsión social puedan
enfocar la intervención y la inversión en las entidades con mayor
riesgo.

Estratégicamente, la aplicación del modelo Lee-Carter
---tradicionalmente empleado para analizar la longevidad y la
sostenibilidad de pensiones--- a la mortalidad infantil, que es un
problema crítico de salud pública, demuestra la versatilidad de las
herramientas actuariales modernas. Esto confirma la visión del CVEA de
que el actuario moderno, con su capacidad para modelar la incertidumbre,
debe colaborar interdisciplinariamente, por ejemplo, con la medicina y
la epidemiología, para abordar los desafíos complejos del país.

\subsection{Objetivo y Contribución}\label{objetivo-y-contribuciuxf3n}

Dentro de este marco de modelación de la incertidumbre, la Tasa de
Mortalidad Infantil (TMI), que cuantifica las defunciones de niños
menores de un año por cada mil nacidos vivos, se erige como un indicador
sociosanitario de relevancia fundamental. La TMI refleja no solo la
condición de salud de la población infantil, sino también el nivel
socioeconómico de una comunidad y la disponibilidad y efectividad de la
atención en materia de salud.

La desagregación de la TMI es vital, ya que las causas y los
determinantes del fallecimiento varían drásticamente según la edad
post-nacimiento. Este estudio se concentra en la clasificación
tripartita establecida por la Organización Mundial de la Salud (OMS) :

\begin{enumerate}
\def\labelenumi{\arabic{enumi}.}
\item
  \textbf{Mortalidad Neonatal (MN):} Se produce desde el nacimiento
  hasta cumplir los 27 días de vida. El riesgo en este periodo es
  fuertemente \textbf{endógeno} , predominando las causas de origen
  perinatal, ligadas a la salud materna, el control del embarazo, la
  calidad de la atención obstétrica y los cuidados inmediatos del recién
  nacido. Las fallas en esta etapa indican debilidades sistémicas en la
  infraestructura sanitaria de alta complejidad.
\item
  \textbf{Mortalidad Postneonatal (MPN):} Se extiende desde los 28 hasta
  los 364 días de vida. Las causas de muerte en este periodo son
  predominantemente \textbf{exógenas} o ``blandas'' , estando más
  relacionadas con las condiciones socioeconómicas, ambientales,
  saneamiento deficiente e infecciones adquiridas (como diarreas o
  trastornos respiratorios agudos).
\item
  \textbf{Mortalidad en la Niñez (N):} Abarca las muertes ocurridas
  entre el primer y el quinto año de vida.
\end{enumerate}

El objetivo principal de este trabajo es modelar y proyectar las tablas
de mortalidad infantil (subdividida en los períodos Neonatal,
Postneonatal y en la Niñez) a nivel de Entidad Federal para el período
2013-2022, basándose en la aplicación del modelo estocástico de
Lee-Carter.

La contribución novedosa del trabajo es doble. Metodológicamente,
establece un estándar riguroso para la cuantificación estocástica de
riesgos de alta resolución en el contexto venezolano. Empíricamente,
proporciona un ``mapa de riesgo'' subnacional sin precedentes, que
permite el diagnóstico de las tendencias de mortalidad regionalmente
divergentes, justificando la transición desde supuestos homogéneos de
riesgo a una gestión basada en la realidad geográfica.

\section{Revisión Literaria}\label{revisiuxf3n-literaria}

\subsection{Metodología de la Revisión y Antecedentes
Actuariales}\label{metodologuxeda-de-la-revisiuxf3n-y-antecedentes-actuariales}

La revisión literaria se estructuró para realizar una síntesis analítica
y sistemática del estado del arte en la modelización de la mortalidad,
demostrando dominio en la literatura demográfica y actuarial. La
estrategia de búsqueda priorizó fuentes de alta confiabilidad en bases
de datos académicas (Scopus, Scielo, Redalyc). Los antecedentes en
proyecciones de mortalidad se clasifican generalmente en modelos
determinísticos (basados en factores de reducción o extrapolaciones
simples) y modelos estocásticos (basados en series de tiempo y
componentes principales).1 El enfoque se centró en la justificación de
por qué los modelos estocásticos, como el Lee-Carter, ofrecen una
alternativa superior para la modelización de riesgos a largo plazo, en
contraste con los métodos tradicionales unidimensionales que solo
consideran la edad y no la evolución temporal.

\subsection{Fundamentos Teóricos de la Mortalidad
Infantil}\label{fundamentos-teuxf3ricos-de-la-mortalidad-infantil}

La Mortalidad Infantil (MI), definida como la muerte de niños menores de
un año, se clasifica rigurosamente en dos grandes componentes debido a
la diferencia en sus causas y determinantes

\begin{enumerate}
\def\labelenumi{\arabic{enumi}.}
\item
  \textbf{Mortalidad Neonatal (MN):} Se produce desde el nacimiento
  hasta cumplir los 27 días de vida. El riesgo en este período es
  fuertemente endógeno y está ligado a causas de origen perinatal,
  incluyendo complicaciones congénitas, de la salud de la madre, el
  control del embarazo, y la calidad de la atención durante el parto y
  los cuidados inmediatos.
\item
  \textbf{Mortalidad Postneonatal (MPN):} Se extiende desde los 28 hasta
  los 364 días de vida. Las causas de muerte en este período son
  predominantemente exógenas o ``blandas'', estando más relacionadas con
  las condiciones socioeconómicas, ambientales, sanitarias deficientes,
  e infecciones (como diarreas o trastornos respiratorios agudos).
\end{enumerate}

Esta distinción es crucial para la modelización, ya que el análisis
desagregado permite a las autoridades sanitarias identificar si las
debilidades se encuentran en el sistema de atención médica primaria (MN)
o en las condiciones de vida y saneamiento (MPN).

\subsection{Modelos Estocásticos de Mortalidad (Familia
Lee-Carter)}\label{modelos-estocuxe1sticos-de-mortalidad-familia-lee-carter}

Desde principios de la década de 1990, los modelos estocásticos han
revolucionado el análisis de la mejora de la mortalidad, destacando el
modelo Lee-Carter (LC) de 1992. El modelo LC es una aproximación
log-lineal que descompone la tasa central de mortalidad \(m_{x,t}\) en
una matriz de dos dimensiones (edad \(x\) y tiempo \(t\)):

\[\log m_{x,t} = \alpha_x + \beta_x \kappa_t + \epsilon_{x,t}\] Donde
\(\alpha_x\) describe el patrón promedio de la mortalidad por edad
(independiente del tiempo), \(\kappa_t\) representa el nivel general de
la mortalidad en el año \(t\) (independiente de la edad) y captura la
tendencia de mejora o deterioro, y \(\beta_x\) mide la sensibilidad de
cada edad \(x\) a los cambios en el nivel general de la mortalidad
\(\kappa_t\).

El éxito del modelo LC radica en su parsimonia y en su capacidad para
transformar un problema de series de tiempo multivariado en uno
univariado, a través de la modelización del índice temporal \(\kappa_t\)
mediante técnicas de series de tiempo (ej. Box-Jenkins). La literatura
empírica internacional ha validado extensivamente la efectividad del
modelo LC en la proyección del equilibrio del sistema de seguridad
social en países como Estados Unidos y Chile, así como en poblaciones
con alta calidad de datos como el G7 y Suecia.

El enfoque metodológico para el análisis de la mortalidad en Venezuela
debe necesariamente abordar la no-uniformidad de la mejora. Al aplicar
el modelo Lee-Carter de manera independiente a cada Entidad Federal, se
permite que cada estado posea su propio índice de mortalidad
(\(\kappa_t\)) y su propia velocidad de cambio (parámetro de deriva,
\(d\)). Esta elección es fundamentalmente un esfuerzo por revelar la
geografía de la mejora de la mortalidad en Venezuela. Si las condiciones
sanitarias, económicas y de seguridad mejoran de manera desigual, el
riesgo futuro también divergerá regionalmente. Utilizar un modelo LC
para cada jurisdicción permite cuantificar esta divergencia.

\subsection{Conclusión de la
Revisión}\label{conclusiuxf3n-de-la-revisiuxf3n}

La revisión confirma que, dada la complejidad generada por las
limitaciones de los modelos determinísticos y la necesidad de modelar la
incertidumbre en un contexto volátil, el enfoque estocástico de
Lee-Carter proporciona el marco metodológico más apropiado. Este marco
es capaz de abordar el doble desafío de generar proyecciones confiables
a largo plazo y capturar la alta heterogeneidad regional de la
mortalidad infantil en Venezuela.

\section{Métodos}\label{muxe9todos}

\subsection{Diseño del Estudio, Fuentes de Datos y
Preprocesamiento}\label{diseuxf1o-del-estudio-fuentes-de-datos-y-preprocesamiento}

El estudio se fundamenta en un análisis de series de tiempo empírico y
riguroso. La principal fuente de datos utilizada para la calibración del
modelo LC provino de los Anuarios de Mortalidad publicados por el
Ministerio del Poder Popular para la Salud (MPPS) . Se compiló
información detallada de las defunciones y la población expuesta al
riesgo para un periodo de 18 años, desde 1995 hasta 2012 . Este lapso
histórico es crucial, ya que provee una base temporal suficiente para la
calibración robusta de los parámetros estocásticos.

Para lograr la alta resolución requerida por el análisis de la
mortalidad infantil, los datos fueron procesados para su desagregación
específica, no solo por Entidad Federal y sexo, sino también por edad
detallada: días (para el periodo Neonatal) y meses (para el periodo
Postneonatal). Este nivel de granularidad es vital para que las
autoridades sanitarias puedan focalizar la intervención en las ventanas
temporales más críticas de la primera infancia.

Un desafío metodológico significativo en la modelación actuarial a nivel
subnacional, especialmente en entidades de baja densidad poblacional
como Amazonas o Delta Amacuro, es el manejo de las tasas centrales de
mortalidad nulas o con valores atípicos, causadas por el registro de un
número reducido de fallecimientos. Para asegurar una calibración robusta
y evitar sesgos derivados de estas tasas nulas, se implementaron
técnicas de interpolación y suavizado de datos, inherentes a las
librerías especializadas del entorno R, basándose en la información de
la misma cohorte de edades similares.

\subsection{Formulación y Calibración del Modelo
Lee-Carter}\label{formulaciuxf3n-y-calibraciuxf3n-del-modelo-lee-carter}

La modelización se centró en la aplicación de la formulación log-lineal
del método Lee-Carter (LC). El modelo se calibra para la tasa central de
mortalidad a la edad \(x\) en el año \(t\), \(m_{x,t}\):

\[\log m_{x,t} = \alpha_x + \beta_x \kappa_t + \epsilon_{x,t} \quad \text{con restricciones: } \sum_t \kappa_t = 0, \sum_x \beta_x = 1\]
Los parámetros fueron estimados bajo el principio de Descomposición de
Valores Singulares (SVD) o métodos de Modelos Lineales Generalizados
(GLM) para obtener \(\hat{\alpha}_x\) (el patrón de edad),
\(\hat{\beta}_x\) (la velocidad de cambio por edad) y \(\hat{\kappa}_t\)
(el índice temporal).

La parte crucial del enfoque estocástico es el modelado y proyección del
índice temporal \(\hat{\kappa}_t\). Este índice, que resume la tendencia
histórica de la mortalidad, se modela como una serie de tiempo. Se
utilizó un modelo de Caminata Aleatoria con Deriva (Random Walk with
Drift), que resultó ser el más apropiado para capturar la tendencia de
mejora a largo plazo:

\[\kappa_t = d + \kappa_{t-1} + e_t\] Donde \(d\) (deriva) es el cambio
anual promedio en el nivel general de la mortalidad, y \(e_t\) es el
término de error. La estimación del parámetro \(d\) es el componente
central de la proyección estocástica, ya que determina la pendiente de
la tendencia futura. Las proyecciones hasta 2022 se calcularon a partir
de esta caminata aleatoria, permitiendo la construcción de intervalos de
confianza alrededor de las tasas proyectadas.

\subsubsection{\texorpdfstring{Proyección y Diagnóstico del Índice
Temporal
(\(\kappa_t\))}{Proyección y Diagnóstico del Índice Temporal (\textbackslash kappa\_t)}}\label{proyecciuxf3n-y-diagnuxf3stico-del-uxedndice-temporal-kappa_t}

La calibración de los parámetros \(\hat{\alpha}_x\), \(\hat{\beta}_x\) y
\(\hat{\kappa}_t\) se realizó a través de métodos de Descomposición de
Valores Singulares (SVD) o Modelos Lineales Generalizados
(GLM).\textsuperscript{1} Una vez estimado el índice temporal
\(\hat{\kappa}_t\), que resume la tendencia histórica de la mortalidad
para cada jurisdicción, se procedió a su modelación estocástica.

El índice \(\hat{\kappa}_t\) se modeló como una Caminata Aleatoria con
Deriva (Random Walk with Drift), un modelo de series de tiempo que ha
demostrado ser el más apropiado para capturar la tendencia de mejora o
deterioro a largo plazo. El componente central de esta modelación es la
estimación del parámetro \(d\), la deriva, la cual determina la
pendiente de la tendencia futura de la mortalidad para la Entidad
Federal específica.\textsuperscript{1}

Las proyecciones se calcularon hasta el año 2022, extendiéndose 10 años
más allá del último dato histórico disponible (2012). La selección de
este horizonte de 10 años es coherente con las recomendaciones
metodológicas que sugieren no proyectar mucho más allá de la mitad del
periodo histórico base (18 años) para mantener la fiabilidad de las
extrapolaciones.

El diagnóstico del ajuste del modelo fue exhaustivo. Se aplicaron
pruebas de diagnóstico estándar para series de tiempo, como las pruebas
de Box-Pierce y Ljung-Box, con el fin de verificar la autocorrelación de
los residuos del modelo. Este rigor estadístico es necesario para
asegurar que el índice temporal \(\kappa_t\) efectivamente capturara la
dinámica histórica de la mortalidad en el tiempo, validando la
adecuación del ajuste para la construcción de intervalos de confianza
alrededor de las tasas proyectadas.

Es fundamental destacar la ventaja de la aplicación regional del modelo
LC. Al ajustar un modelo independiente a cada Entidad Federal, la gran
varianza asociada a la heterogeneidad geográfica del riesgo no se carga
en el término de error \(\epsilon_{x,t}\) del modelo. Esto resulta en
una reducción de la varianza de los residuos y, por consiguiente, en
proyecciones puntuales más confiables a nivel local, con intervalos de
confianza más estrechos, lo cual es de máxima utilidad para la
planificación subnacional.

\section{Resultados}\label{resultados}

Los resultados empíricos históricos (1995-2012) y las proyecciones
estocásticas (2013-2022) generadas por el modelo Lee-Carter confirmaron
una profunda y persistente heterogeneidad en el riesgo de mortalidad
infantil a nivel subnacional, desvirtuando cualquier supuesto de
uniformidad de riesgo en el país.\textsuperscript{1}

\subsection{Concentración del Riesgo: Mortalidad Neonatal (MN: 0-27
Días)}\label{concentraciuxf3n-del-riesgo-mortalidad-neonatal-mn-0-27-duxedas}

El análisis de la Mortalidad Neonatal (MN) es crucial, ya que el mayor
peso de las defunciones infantiles se concentra antes de cumplir los 28
días de vida.

\subsubsection{La Geografía del Riesgo Extremo: El Factor Zulia y
Trujillo}\label{la-geografuxeda-del-riesgo-extremo-el-factor-zulia-y-trujillo}

El estado Zulia se ubicó consistentemente como la entidad con las tasas
de mortalidad más altas en el periodo MN. Este fenómeno, denominado el
``Factor Zulia'', sugiere una causalidad macro-estructural, ligada a
deficiencias crónicas y profundas en el sistema de salud regional.

\begin{itemize}
\tightlist
\item
  \textbf{Zulia: Epicentro Consistente de Riesgo:} El estado Zulia
  presentó una tasa promedio de mortalidad neonatal de
  \(12.16 \text{‰}\) entre 1985 y 2007. No obstante, el pico de riesgo
  registrado en el periodo de estudio 1995-2012 se dio en el año
  \textbf{2007}, con una tasa máxima de \(57.19\) por cada \(1,000\)
  nacidos vivos.
\end{itemize}

\phantomsection\label{fig:zulia-1}
\pandocbounded{\includegraphics[keepaspectratio]{figure/zulia1.jpg}}
Mortalidad Neonatal Masculina Zulia

\phantomsection\label{fig:zulia-2}
\pandocbounded{\includegraphics[keepaspectratio]{figure/zulia2.jpg}}
Mortalidad Neonatal Masculina Zulia

\begin{itemize}
\tightlist
\item
  \textbf{Aragua (Riesgo Histórico Elevado):} El estado Aragua
  históricamente mostró un riesgo neonatal elevado, con una tasa
  promedio de \(11.23 \text{‰}\) para el periodo
  1985-2007.\textsuperscript{1} En el lapso estudiado, se registró una
  alta tasa de \(9.19 \text{‰}\) a los 0 días de edad en 1995.
\end{itemize}

\phantomsection\label{fig:Aragua-1}
\pandocbounded{\includegraphics[keepaspectratio]{figure/aragua1.jpg}}
Mortalidad Neonatal Masculina Aragua

\phantomsection\label{fig:Aragua-2}
\pandocbounded{\includegraphics[keepaspectratio]{figure/aragua2.jpg}}
Mortalidad Neonatal Masculina Aragua

\begin{itemize}
\tightlist
\item
  \textbf{Trujillo: La Advertencia de Deterioro Proyectado:}
  Históricamente, Trujillo alcanzó una tasa promedio de
  \(10.96 \text{‰}\) entre 1985 y 2007. A pesar de que la tendencia
  general proyectada por el modelo LC para la mayoría de los estados
  apuntaba hacia un descenso progresivo de las tasas de mortalidad
  neonatal, se identificó un resultado proyectado altamente
  significativo y preocupante en el estado Trujillo. El modelo
  Lee-Carter pronosticó un \textbf{ascenso considerable y sostenido en
  la mortalidad neonatal a partir de los años 2014-2016}. Este
  incremento proyectado constituyó un \textbf{punto de quiebre} en la
  tendencia histórica de reducción. El ascenso proyectado se dispara a
  \textbf{más de 10 puntos por encima del promedio histórico} de las
  tasas observadas en los años inmediatamente anteriores. La tasa, que
  era de \(12.3 \text{‰}\) en 2013, se elevó a \(14.2 \text{‰}\) en
  2014.
\end{itemize}

\phantomsection\label{fig:Trujillo-1}
\pandocbounded{\includegraphics[keepaspectratio]{figure/Trujillo1.jpg}}
Mortalidad Neonatal Masculina Trujillo

\phantomsection\label{fig:Trujillo-2}
\pandocbounded{\includegraphics[keepaspectratio]{figure/Trujillo2.jpg}}
Mortalidad Neonatal Masculina Trujillo

La aparición de una deriva (\(d\)) positiva en el índice temporal
\(\kappa_t\) de Trujillo, que se traduce en un ascenso en el riesgo
proyectado, constituye una \textbf{advertencia de deterioro agudo del
sistema de salud regional}.

\subsubsection{Patrones de Resiliencia y Disparidad (Miranda, Falcón y
Sucre)}\label{patrones-de-resiliencia-y-disparidad-miranda-falcuxf3n-y-sucre}

El análisis también identificó a las Entidades Federales que exhibieron
las tasas más bajas de mortalidad neonatal histórica. La diferencia
entre el riesgo máximo y mínimo subraya la disparidad de condiciones de
salud en el territorio.

\begin{itemize}
\tightlist
\item
  \textbf{Miranda:} Mantuvo la tasa máxima más baja de todo el país
  dentro de las entidades analizadas, registrando un máximo de \(2.88\)
  por cada \(1,000\) nacidos vivos en 1996. Su tasa promedio histórica
  fue de \(7.54 \text{‰}\).
\end{itemize}

\phantomsection\label{fig:Miranda-1}
\pandocbounded{\includegraphics[keepaspectratio]{figure/Miranda1.jpg}}
Mortalidad Neonatal Masculina Miranda

\phantomsection\label{fig:Miranda-2}
\pandocbounded{\includegraphics[keepaspectratio]{figure/Miranda2.jpg}}
Mortalidad Neonatal Masculina Miranda

\begin{itemize}
\tightlist
\item
  \textbf{Falcón:} Presentó una tasa promedio de \(7.50 \text{‰}\).
\end{itemize}

\phantomsection\label{fig:Falcon-1}
\pandocbounded{\includegraphics[keepaspectratio]{figure/Falcon1.jpg}}
Mortalidad Neonatal Masculina Falcon

\phantomsection\label{fig:Falcon-2}
\pandocbounded{\includegraphics[keepaspectratio]{figure/Falcon2.jpg}}
Mortalidad Neonatal Masculina Falcon

\begin{itemize}
\tightlist
\item
  \textbf{Sucre:} Reportó la tasa promedio más baja en el periodo de
  referencia (\(7.49 \text{‰}\)).
\end{itemize}

\phantomsection\label{fig:Sucre-1}
\pandocbounded{\includegraphics[keepaspectratio]{figure/Sucre1.jpg}}
Mortalidad Neonatal Masculina Sucre

\phantomsection\label{fig:Sucre-2}
\pandocbounded{\includegraphics[keepaspectratio]{figure/Sucre2.jpg}}
Mortalidad Neonatal Masculina Sucre

La comparación entre los extremos es ilustrativa.La disparidad entre el
riesgo máximo histórico de Zulia (\(57.19 \text{‰}\)) y el mínimo de
Miranda (\(2.88 \text{‰}\)) establece un \textbf{ratio de disparidad de
aproximadamente} \(20:1\).

\subsubsection{Análisis Descriptivo General y Causas
Subyacentes}\label{anuxe1lisis-descriptivo-general-y-causas-subyacentes}

El análisis descriptivo general del periodo Neonatal mostró un patrón
consistente: el mayor peso de las defunciones se concentra antes de
cumplir los 28 días. Este riesgo elevado, particularmente en las
primeras 24 horas y durante los primeros 27 días, se debe
fundamentalmente a \textbf{afecciones originadas en el periodo
perinatal}. Esto incluye bajo peso al nacer, prematuridad,
complicaciones durante el parto y malformaciones congénitas.

Las causas de defunción confirman que:

\begin{itemize}
\item
  \textbf{Causas Primarias (Endógenas):} Las afecciones originadas en el
  periodo perinatal son la principal causa de mortalidad neonatal a lo
  largo de los años.
\item
  \textbf{Causas Secundarias y Terciarias:} Las enfermedades del sistema
  respiratorio y ciertas enfermedades infecciosas y parasitarias son la
  segunda y tercera causa más importante, respectivamente.
\item
  \textbf{Diferencial de Género:} Un patrón constante observado en la
  mayoría de las entidades federales fue que el género \textbf{masculino
  exhibió consistentemente una mayor probabilidad de fallecer} en el
  periodo neonatal que el femenino. Esta sobremortalidad biológica se
  concentra de manera más intensa en las primeras 24 horas de vida del
  neonato.
\end{itemize}

Esto implica que la debilidad no reside primariamente en el saneamiento
o el ambiente post-neonatal, sino en la \textbf{calidad deficiente del
control prenatal, la infraestructura obstétrica y los cuidados
intensivos neonatales}.

\subsection{Transición del Riesgo: Mortalidad Postneonatal (MPN) y en la
Niñez
(N)}\label{transiciuxf3n-del-riesgo-mortalidad-postneonatal-mpn-y-en-la-niuxf1ez-n}

El riesgo de mortalidad evoluciona con la edad. A partir de los 28 días,
el patrón causal de las defunciones transita de ser predominantemente
endógeno a \textbf{exógeno}. En esta etapa, el riesgo está impulsado por
factores ambientales, socioeconómicos y de acceso a la atención básica,
tales como infecciones, diarreas, deshidratación y trastornos
respiratorios agudos.

\subsubsection{Alto Riesgo Persistente en la Periferia y Anomalías de
Género}\label{alto-riesgo-persistente-en-la-periferia-y-anomaluxedas-de-guxe9nero}

Aunque Zulia mantiene el riesgo más alto, otras entidades de baja
densidad poblacional y ubicaciones remotas muestran una vulnerabilidad
estructural persistente en los periodos MPN y en la Niñez.

\begin{itemize}
\tightlist
\item
  \textbf{Mortalidad Postneonatal (MPN):} Zulia se mantuvo como líder en
  la tasa máxima histórica, con \(23.91 \text{‰}\) en 2007. Delta
  Amacuro se ubicó como el segundo estado con mayor tasa, alcanzando
  \(8.94 \text{‰}\) a los 4 meses de edad en 1998. Amazonas fue el
  tercer estado con mayor riesgo de MPN.
\end{itemize}

\phantomsection\label{fig:Zulia-3}
\pandocbounded{\includegraphics[keepaspectratio]{figure/Zulia3.jpg}}
Mortalidad Neonatal Masculina Zulia

\phantomsection\label{fig:Zulia-4}
\pandocbounded{\includegraphics[keepaspectratio]{figure/Zulia4.jpg}}
Mortalidad Neonatal Masculina Zulia

\phantomsection\label{fig:Delta-1}
\pandocbounded{\includegraphics[keepaspectratio]{figure/Delta1.jpg}}
Mortalidad Neonatal Masculina Delta

\phantomsection\label{fig:Delta-2}
\pandocbounded{\includegraphics[keepaspectratio]{figure/Delta2.jpg}}
Mortalidad Neonatal Masculina Delta

\phantomsection\label{fig:Amazonas-1}
\pandocbounded{\includegraphics[keepaspectratio]{figure/Amazonas1.jpg}}
Mortalidad Neonatal Masculina Amazonas

\phantomsection\label{fig:Amazonas-2}
\pandocbounded{\includegraphics[keepaspectratio]{figure/Amazonas2.jpg}}
Mortalidad Neonatal Masculina Amazonas

\begin{itemize}
\item
  \textbf{Anomalía de Género en Amazonas (MPN):} En el periodo
  Postneonatal, se identificó un patrón de mortalidad atípico en el
  estado Amazonas, un fenómeno que invierte la tendencia biológica
  general de sobremortalidad masculina en la primera infancia. En
  Amazonas, el índice de \textbf{mortalidad femenina superó al
  masculino} en el periodo Postneonatal. Este patrón inusual sugiere la
  existencia de \textbf{factores causales específicos de la región,
  posiblemente de índole socio-cultural o ambiental}.
\item
  \textbf{Mortalidad en la Niñez (N):} Zulia volvió a registrar la tasa
  más alta con \(46.01 \text{‰}\) en 2004.\textsuperscript{1} Le
  siguieron Amazonas, con \(17.41 \text{‰}\) en 1996, y Delta Amacuro,
  con \(11.90 \text{‰}\) en 1997.
\end{itemize}

\phantomsection\label{fig:Zulia-5}
\pandocbounded{\includegraphics[keepaspectratio]{figure/Zulia5.jpg}}
Mortalidad Neonatal Masculina Zulia

\phantomsection\label{fig:Zulia-6}
\pandocbounded{\includegraphics[keepaspectratio]{figure/Zulia6.jpg}}
Mortalidad Neonatal Masculina Zulia

\phantomsection\label{fig:Delta-3}
\pandocbounded{\includegraphics[keepaspectratio]{figure/Delta3.jpg}}
Mortalidad Neonatal Masculina Delta

\phantomsection\label{fig:Delta-4}
\pandocbounded{\includegraphics[keepaspectratio]{figure/Delta4.jpg}}
Mortalidad Neonatal Masculina Delta

\phantomsection\label{fig:Amazonas-3}
\pandocbounded{\includegraphics[keepaspectratio]{figure/Amazonas3.jpg}}
Mortalidad Neonatal Masculina Amazonas

\phantomsection\label{fig:Amazonas-4}
\pandocbounded{\includegraphics[keepaspectratio]{figure/Amazonas4.jpg}}
Mortalidad Neonatal Masculina Amazonas

La persistencia de entidades como Delta Amacuro y Amazonas en el grupo
de alto riesgo a lo largo de 18 años de datos históricos es un hallazgo
robusto. Esto indica que la falta de acceso geográfico a la atención
primaria de salud, el saneamiento y la calidad de vida son factores de
riesgo crónicos que el promedio nacional de mejora nunca ha logrado
mitigar.

\subsection{Síntesis de la Heterogeneidad Geográfica Histórica
(1995-2012)}\label{suxedntesis-de-la-heterogeneidad-geogruxe1fica-histuxf3rica-1995-2012}

La magnitud de la heterogeneidad geográfica del riesgo se consolida en
la siguiente tabla que resume los extremos históricos observados:

\begin{longtable}[]{@{}
  >{\raggedright\arraybackslash}p{(\linewidth - 12\tabcolsep) * \real{0.1429}}
  >{\raggedright\arraybackslash}p{(\linewidth - 12\tabcolsep) * \real{0.1429}}
  >{\raggedright\arraybackslash}p{(\linewidth - 12\tabcolsep) * \real{0.1429}}
  >{\raggedright\arraybackslash}p{(\linewidth - 12\tabcolsep) * \real{0.1429}}
  >{\raggedright\arraybackslash}p{(\linewidth - 12\tabcolsep) * \real{0.1429}}
  >{\raggedright\arraybackslash}p{(\linewidth - 12\tabcolsep) * \real{0.1429}}
  >{\raggedright\arraybackslash}p{(\linewidth - 12\tabcolsep) * \real{0.1429}}@{}}
\toprule\noalign{}
\endhead
\bottomrule\noalign{}
\endlastfoot
\textbf{Período} & \textbf{Entidad de Máximo Riesgo} & \textbf{Tasa
Máxima (por 1,000 NV)} & \textbf{Año Pico} & \textbf{Entidad de Mínimo
Riesgo} & \textbf{Tasa Mínima Histórica (por 1,000 NV)} & \textbf{Ratio
de Disparidad (Máx/Mín)} \\
Neonatal (MN) & Zulia & 57.19 & 2007 & Miranda & 2.88 &
\(\approx 20:1\) \\
Postneonatal (MPN) & Zulia & 23.91 & 2007 & Sucre & 1.36 &
\(\approx 18:1\) \\
Niñez (N) & Zulia & 46.01 & 2004 & Anzoátegui & 1.60 &
\(\approx 29:1\) \\
\end{longtable}

La disparidad extrema en el riesgo, con ratios de hasta \(29:1\) (Zulia
vs.~Anzoátegui en la Niñez), es la evidencia fundamental de la
\textbf{ineficiencia de la tarificación actuarial no regionalizada} y la
necesidad de integrar la geografía como un factor de riesgo crucial en
el diseño de productos y políticas.

\section{Discusión y Conclusiones}\label{discusiuxf3n-y-conclusiones}

\subsection{Interpretación de los Hallazgos Centrales y Diálogo con el
Estado del
Arte}\label{interpretaciuxf3n-de-los-hallazgos-centrales-y-diuxe1logo-con-el-estado-del-arte}

Los resultados confirman que el riesgo de mortalidad infantil en
Venezuela está fuertemente concentrado y no es homogéneo. El predominio
del riesgo en la Mortalidad Neonatal (0-27 días) sobre la Postneonatal
sugiere que los desafíos más acuciantes se encuentran en la esfera de la
atención médica directa: control prenatal, infraestructura obstétrica y
cuidados intensivos neonatales. Esto se alinea con la controversia
epidemiológica que vincula la MN con fallas sistémicas en el ámbito
sanitario. El fenómeno del Factor Zulia, donde la entidad concentra el
riesgo más alto en todas las etapas de la mortalidad infantil, indica
que la causalidad es probablemente macro-estructural, ligada a
deficiencias crónicas y profundas en el sistema de salud regional. Esta
vulnerabilidad estructural magnifica el impacto de cualquier shock
adverso, como el pico de 2007. La Advertencia de Trujillo es, quizás, el
hallazgo más relevante desde la perspectiva de la política pública. El
ascenso proyectado en la mortalidad neonatal de Trujillo, en contraste
con las tendencias nacionales históricas de mejora, es una señal
inequívoca de una divergencia regional crítica.1 Este resultado tiene
dos implicaciones fundamentales: Demuestra que la suposición de
tendencias homogéneas, característica de los modelos determinísticos, es
inadecuada para la planificación en Venezuela. Valida la necesidad de
utilizar modelos estocásticos y granularmente aplicados para identificar
estos ``puntos de inflexión'' o dinámicas de deterioro regional antes de
que se reflejen en los datos nacionales agregados.

\subsection{Limitaciones del Estudio y
Sesgos}\label{limitaciones-del-estudio-y-sesgos}

El principal factor limitante de este análisis es el período de
referencia, que culmina en 2012.1 La modelación estocástica de
Lee-Carter se basa intrínsecamente en la extrapolación de la deriva
histórica (\(d\)). Por lo tanto, si las condiciones socioeconómicas y
sanitarias han experimentado un deterioro significativo y sostenido
posterior a 2012 (como ha sido documentado por otros estudios), el
modelo podría subestimar el riesgo actual, dado que la tendencia
histórica incluida es predominantemente de mejora o estabilidad.
Adicionalmente, se reconoce el potencial sesgo de volatilidad en los
estados de baja población, como Delta Amacuro y Amazonas. En estas
entidades, un número reducido de fallecimientos puede generar tasas
extremas debido a la escasa población expuesta al riesgo. No obstante,
la persistencia de estas regiones en el grupo de alto riesgo a lo largo
de 18 años de datos históricos es un hallazgo robusto que no debe ser
desestimado.

\subsection{Implicaciones y Futuras
Direcciones}\label{implicaciones-y-futuras-direcciones}

\textbf{Implicaciones de Política Pública}

La evidencia generada por el modelado estocástico subnacional ofrece
recomendaciones claras para una intervención efectiva:

\begin{enumerate}
\def\labelenumi{\arabic{enumi}.}
\tightlist
\item
  \textbf{Focalización Geográfica Prioritaria:} La asignación
  presupuestaria y la inversión en infraestructura de salud deben
  concentrarse prioritariamente en las regiones de riesgo extremo,
  identificadas consistentemente como Zulia, Amazonas y Delta Amacuro.
\item
  \textbf{Focalización Temporal en la Ventana Neonatal:} Las estrategias
  de intervención deben maximizar el impacto en el período de 0 a 27
  días de vida. Esto implica el fortalecimiento urgente del control
  prenatal, la mejora de la calidad de la atención obstétrica y la
  dotación de unidades de neonatología funcionales. La reducción del
  riesgo neonatal tendrá el mayor efecto apalancamiento en la reducción
  de la TMI general.
\end{enumerate}

\textbf{Implicaciones para la Práctica Actuarial}

Los hallazgos de este trabajo son cruciales para la modernización del
sector de seguros y pensiones: Reforma de Bases Técnicas Actuariales: Se
hace un llamado a los entes reguladores (como Sudeaseg) y a la industria
para que integren las proyecciones estocásticas del riesgo infantil en
sus cálculos. La dependencia de bases estáticas u obsoletas ignora el
riesgo real y la volatilidad, llevando a una estimación incorrecta de
las primas y las reservas.1 Tarificación Basada en Riesgo Regional: La
clara evidencia de la heterogeneidad geográfica del riesgo (un riesgo
veinte veces mayor en Zulia que en Miranda en el pico histórico) exige
la creación de factores de ajuste regionales para la tarificación de
seguros de vida y salud infantil. Esto permitirá una tarificación más
justa y ajustada a la realidad del riesgo geográfico local.1

\subsection{Conclusión Final}\label{conclusiuxf3n-final}

Este estudio de modelación y proyección de la mortalidad infantil
demuestra cómo el rigor de la modelación estocástica basada en el modelo
Lee-Carter puede aplicarse eficazmente a desafíos específicos de salud
pública, proveyendo un mapa de riesgo y de incertidumbre que es
indispensable en el contexto volátil venezolano. Se concluye que la
mitigación efectiva de la mortalidad infantil y la optimización de los
recursos de salud requieren, de manera fundamental, una planificación
estratégica que abandone la homogeneidad y se base en una estricta
estratificación del riesgo, tanto geográfica como temporal, priorizando
la intervención en la fase neonatal y en las entidades federales
identificadas con riesgo extremo.

\section*{Agradecimientos}\label{agradecimientos}
\addcontentsline{toc}{section}{Agradecimientos}

\section*{Referencias}\label{referencias}
\addcontentsline{toc}{section}{Referencias}

La lista de referencias se generará automáticamente aquí a partir del
archivo \texttt{references.bib} y las citas usadas en el texto. No es
necesario escribir nada en esta sección.

\emph{Ejemplos de Referencias Generadas (Formato APA):}

\begin{itemize}
\item
  Akaike, H. (1974). A new look at the statistical model identification.
  \emph{IEEE Transactions on Automatic Control}, \emph{19}(6), 716--723.
\item
  Breusch, T. S., \& Pagan, A. R. (1979). A simple test for
  heteroskedasticity and random coefficient variation.
  \emph{Econometrica}, \emph{47}(5), 1287--1294.
\end{itemize}

\begin{center}\rule{0.5\linewidth}{0.5pt}\end{center}

\newpage

\section*{Comentarios Finales y
Recomendaciones}\label{comentarios-finales-y-recomendaciones}
\addcontentsline{toc}{section}{Comentarios Finales y Recomendaciones}

Esta plantilla ha sido creada con Quarto, un sistema de publicación
científica de código abierto que integra texto y código para producir
documentos reproducibles y de alta calidad.

La adopción de Quarto y la exigencia de la nueva estructura I-RL-M-R-D
se complementan con un estricto enfoque en la reproducibilidad. Los
modelos actuariales, dada su función crítica en la toma de decisiones
financieras y de riesgo, deben ser totalmente transparentes. La
metodología de Quarto permite que el código fuente sea parte integral
del manuscrito, lo que constituye el estándar de oro para la
verificación por pares.




\end{document}
