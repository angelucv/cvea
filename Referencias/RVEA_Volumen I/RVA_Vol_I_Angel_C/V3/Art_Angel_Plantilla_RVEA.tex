% Options for packages loaded elsewhere
% Options for packages loaded elsewhere
\PassOptionsToPackage{unicode}{hyperref}
\PassOptionsToPackage{hyphens}{url}
\PassOptionsToPackage{dvipsnames,svgnames,x11names}{xcolor}
%
\documentclass[
  11pt,
  a4paper,
]{article}
\usepackage{xcolor}
\usepackage{amsmath,amssymb}
\setcounter{secnumdepth}{-\maxdimen} % remove section numbering
\usepackage{iftex}
\ifPDFTeX
  \usepackage[T1]{fontenc}
  \usepackage[utf8]{inputenc}
  \usepackage{textcomp} % provide euro and other symbols
\else % if luatex or xetex
  \usepackage{unicode-math} % this also loads fontspec
  \defaultfontfeatures{Scale=MatchLowercase}
  \defaultfontfeatures[\rmfamily]{Ligatures=TeX,Scale=1}
\fi
\usepackage{lmodern}
\ifPDFTeX\else
  % xetex/luatex font selection
\fi
% Use upquote if available, for straight quotes in verbatim environments
\IfFileExists{upquote.sty}{\usepackage{upquote}}{}
\IfFileExists{microtype.sty}{% use microtype if available
  \usepackage[]{microtype}
  \UseMicrotypeSet[protrusion]{basicmath} % disable protrusion for tt fonts
}{}
\usepackage{setspace}
\makeatletter
\@ifundefined{KOMAClassName}{% if non-KOMA class
  \IfFileExists{parskip.sty}{%
    \usepackage{parskip}
  }{% else
    \setlength{\parindent}{0pt}
    \setlength{\parskip}{6pt plus 2pt minus 1pt}}
}{% if KOMA class
  \KOMAoptions{parskip=half}}
\makeatother
% Make \paragraph and \subparagraph free-standing
\makeatletter
\ifx\paragraph\undefined\else
  \let\oldparagraph\paragraph
  \renewcommand{\paragraph}{
    \@ifstar
      \xxxParagraphStar
      \xxxParagraphNoStar
  }
  \newcommand{\xxxParagraphStar}[1]{\oldparagraph*{#1}\mbox{}}
  \newcommand{\xxxParagraphNoStar}[1]{\oldparagraph{#1}\mbox{}}
\fi
\ifx\subparagraph\undefined\else
  \let\oldsubparagraph\subparagraph
  \renewcommand{\subparagraph}{
    \@ifstar
      \xxxSubParagraphStar
      \xxxSubParagraphNoStar
  }
  \newcommand{\xxxSubParagraphStar}[1]{\oldsubparagraph*{#1}\mbox{}}
  \newcommand{\xxxSubParagraphNoStar}[1]{\oldsubparagraph{#1}\mbox{}}
\fi
\makeatother


\usepackage{longtable,booktabs,array}
\usepackage{calc} % for calculating minipage widths
% Correct order of tables after \paragraph or \subparagraph
\usepackage{etoolbox}
\makeatletter
\patchcmd\longtable{\par}{\if@noskipsec\mbox{}\fi\par}{}{}
\makeatother
% Allow footnotes in longtable head/foot
\IfFileExists{footnotehyper.sty}{\usepackage{footnotehyper}}{\usepackage{footnote}}
\makesavenoteenv{longtable}
\usepackage{graphicx}
\makeatletter
\newsavebox\pandoc@box
\newcommand*\pandocbounded[1]{% scales image to fit in text height/width
  \sbox\pandoc@box{#1}%
  \Gscale@div\@tempa{\textheight}{\dimexpr\ht\pandoc@box+\dp\pandoc@box\relax}%
  \Gscale@div\@tempb{\linewidth}{\wd\pandoc@box}%
  \ifdim\@tempb\p@<\@tempa\p@\let\@tempa\@tempb\fi% select the smaller of both
  \ifdim\@tempa\p@<\p@\scalebox{\@tempa}{\usebox\pandoc@box}%
  \else\usebox{\pandoc@box}%
  \fi%
}
% Set default figure placement to htbp
\def\fps@figure{htbp}
\makeatother





\setlength{\emergencystretch}{3em} % prevent overfull lines

\providecommand{\tightlist}{%
  \setlength{\itemsep}{0pt}\setlength{\parskip}{0pt}}



 


\usepackage{fancyhdr}
\usepackage{titling}
\usepackage{fvextra}
\usepackage{graphicx}
\usepackage{xcolor}

% --- CONFIGURACIÓN DE PORTADA (LOGO Y ESPACIADO) ---
\pretitle{
\begin{center}
\IfFileExists{cvea-logo.png}{
\includegraphics[width=8cm]{cvea-logo.png}}{\vspace*{1cm}\fbox{\textbf{Logo Institucional (cvea-logo.png)}}}
% Añadimos un espacio vertical explícito de 1cm entre logo y título
\par\vspace{1cm}
\LARGE\bfseries
}
\posttitle{\end{center}}

% --- CONFIGURACIÓN DE ENCABEZADO DINÁMICO ---
\pagestyle{fancy}
\fancyhf{}
\setlength{\headheight}{1.5cm}
\setlength{\headsep}{25pt}

% Lado Izquierdo: Logo pequeño
\fancyhead[L]{
\IfFileExists{cvea-logo.png}{
\includegraphics[height=1cm]{cvea-logo.png}
}{
\small [Logo]
}
}

% Lado Derecho: Nombre de la sección actual
\fancyhead[R]{
\small\itshape\nouppercase{\leftmark}
}

\fancyfoot[C]{\thepage}
\renewcommand{\headrulewidth}{0.4pt}

% Ajuste para la primera página (estilo plain)
\fancypagestyle{plain}{
\fancyhf{}
\renewcommand{\headrulewidth}{0pt}
\fancyfoot[C]{\thepage}
}

\fvset{breaklines=true, breakanywhere=true}
\makeatletter
\@ifpackageloaded{caption}{}{\usepackage{caption}}
\AtBeginDocument{%
\ifdefined\contentsname
  \renewcommand*\contentsname{Table of contents}
\else
  \newcommand\contentsname{Table of contents}
\fi
\ifdefined\listfigurename
  \renewcommand*\listfigurename{List of Figures}
\else
  \newcommand\listfigurename{List of Figures}
\fi
\ifdefined\listtablename
  \renewcommand*\listtablename{List of Tables}
\else
  \newcommand\listtablename{List of Tables}
\fi
\ifdefined\figurename
  \renewcommand*\figurename{Figure}
\else
  \newcommand\figurename{Figure}
\fi
\ifdefined\tablename
  \renewcommand*\tablename{Table}
\else
  \newcommand\tablename{Table}
\fi
}
\@ifpackageloaded{float}{}{\usepackage{float}}
\floatstyle{ruled}
\@ifundefined{c@chapter}{\newfloat{codelisting}{h}{lop}}{\newfloat{codelisting}{h}{lop}[chapter]}
\floatname{codelisting}{Listing}
\newcommand*\listoflistings{\listof{codelisting}{List of Listings}}
\makeatother
\makeatletter
\makeatother
\makeatletter
\@ifpackageloaded{caption}{}{\usepackage{caption}}
\@ifpackageloaded{subcaption}{}{\usepackage{subcaption}}
\makeatother
\usepackage{bookmark}
\IfFileExists{xurl.sty}{\usepackage{xurl}}{} % add URL line breaks if available
\urlstyle{same}
\hypersetup{
  pdftitle={Análisis Espacio-Temporal de la Mortalidad en Venezuela (1996-2016)},
  pdfauthor={Prof.~Angel Colmenares},
  pdfkeywords={Mortalidad, Venezuela, INLA, SMR, Análisis Espacial},
  colorlinks=true,
  linkcolor={blue},
  filecolor={Maroon},
  citecolor={Blue},
  urlcolor={Blue},
  pdfcreator={LaTeX via pandoc}}


\title{Análisis Espacio-Temporal de la Mortalidad en Venezuela
(1996-2016)}
\usepackage{etoolbox}
\makeatletter
\providecommand{\subtitle}[1]{% add subtitle to \maketitle
  \apptocmd{\@title}{\par {\large #1 \par}}{}{}
}
\makeatother
\subtitle{Un enfoque de Riesgo Bayesiano y Autocorrelación Espacial a
nivel municipal}
\author{Prof.~Angel Colmenares}
\date{}
\begin{document}
\maketitle
\begin{abstract}
\textbf{Resumen:} La evaluación de la mortalidad sub-nacional es crítica
para la solvencia de sistemas de previsión social y la tarificación de
seguros de vida. Este estudio analiza la mortalidad en los 335
municipios de Venezuela (1996-2016) combinando normalización geográfica
(GADM), clusters de K-means, índices de autocorrelación espacial de
Moran y un modelo jerárquico Bayesiano Besag-York-Mollié (BYM) mediante
INLA. Se identificó una tendencia nacional de incremento del riesgo del
1.4\% anual, con clusters de riesgo persistentes en el arco central y
zonas fronterizas. La interpolación IDW del riesgo bayesiano revela una
superficie de mortalidad heterogénea que invalida el uso de tablas de
mortalidad únicas nacionales. Existe un exceso de mortalidad crónico en
regiones específicas que requiere ajustes actuariales territorializados.

\textbf{Abstract:} Sub-national mortality assessment is critical for the
solvency of social security systems and life insurance pricing. This
study analyzes mortality across Venezuela's 335 municipalities
(1996-2016) combining geographic normalization (GADM), K-means
clustering, Moran's spatial autocorrelation indices, and a Bayesian
Besag-York-Mollié (BYM) hierarchical model via INLA. A national
increasing risk trend of 1.4\% per year was identified, with persistent
risk clusters in the central arc and border areas. IDW interpolation of
Bayesian risk reveals a heterogeneous mortality surface that invalidates
the use of single national mortality tables. There is chronic excess
mortality in specific regions requiring territorialized actuarial
adjustments.
\end{abstract}


\setstretch{1.5}
\section{Introducción}\label{introducciuxf3n}

La dinámica demográfica de Venezuela ha experimentado transformaciones
estructurales profundas en las últimas dos décadas. El análisis de la
mortalidad, entendido como el componente esencial del riesgo actuarial y
la base de la seguridad social, requiere una lente geográfica de alta
resolución que permita identificar desigualdades territoriales que las
métricas nacionales agregadas suelen ocultar. Este estudio se sitúa en
un contexto crítico para la ciencia actuarial venezolana: la necesidad
de modernizar las bases técnicas que sustentan los sistemas de previsión
social y la tarificación de seguros de vida.

Durante décadas, la práctica actuarial en el país ha dependido de tablas
de mortalidad estáticas, como las desarrolladas por el Profesor Víctor
Masjuán en los años 60 o la tabla CSO 1980, las cuales presentan un
desfase significativo frente a la realidad actual. La investigación de
Daylin Moreno (2018) ya ha advertido sobre el ``pecado original'' del
sistema de pensiones: una subestimación sistemática de la supervivencia
que ha erosionado la solvencia del Instituto Venezolano de los Seguros
Sociales (IVSS). En este escenario, la transición hacia modelos
dinámicos y espaciales no es solo una mejora metodológica, sino un
imperativo de solvencia institucional.

El presente artículo aborda el período 1996-2016, un lapso que captura
la evolución de la mortalidad antes y durante cambios socioeconómicos de
gran calado. El objetivo central es determinar si la distribución de la
mortalidad en los 335 municipios de Venezuela responde a un patrón
puramente aleatorio o si existen estructuras de riesgo espaciales
persistentes. Para ello, se emplea una estrategia que transita desde la
exploración por clusters de K-means hasta la modelización bayesiana
avanzada mediante el modelo Besag-York-Mollié (BYM), filtrando el ruido
aleatorio inherente a las áreas con poblaciones pequeñas. Este trabajo
se integra en el primer volumen de la Revista Venezolana de Actuariado
(RVEA) como un aporte fundamental del Centro Venezolano de Estudios
Actuariales (CVEA).

\section{Revisión Literaria}\label{revisiuxf3n-literaria}

La fundamentación teórica de la epidemiología espacial y el análisis
actuarial territorial se erige sobre la Primera Ley de la Geografía de
Tobler (1970), la cual establece que la proximidad física implica una
mayor correlación entre observaciones. En la práctica actuarial, la
Razón de Mortalidad Estandarizada (SMR) ha sido la métrica convencional
para normalizar la comparación entre unidades geográficas con
estructuras demográficas disímiles (Clayton \& Kaldor, 1987; Waller \&
Gotway, 2004). No obstante, autores como Besag, York y Mollié (1991) han
demostrado que en áreas con baja densidad poblacional ---problema común
en la desagregación municipal--- las tasas crudas y los SMR exhiben una
alta inestabilidad, donde eventos fortuitos pueden distorsionar
significativamente el perfil de riesgo.

Para mitigar este ``ruido de áreas pequeñas'', la literatura
contemporánea favorece el uso de modelos jerárquicos bayesianos,
particularmente el modelo Besag-York-Mollié (BYM). Este enfoque permite
``tomar prestada'' información de los municipios vecinos para producir
estimaciones del riesgo relativo (RR) suavizadas y estadísticamente
robustas (Lawson, 2018). La implementación de estos modelos ha sido
revolucionada por la técnica de Aproximación de Laplace Anidada
Integrada (INLA), propuesta por Rue, Martino y Chopin (2009), la cual
ofrece una alternativa computacionalmente eficiente frente a los
algoritmos de Cadenas de Markov Monte Carlo (MCMC), permitiendo el
procesamiento de grandes volúmenes de datos espacio-temporales
(Blangiardo \& Cameletti, 2015).

En el ámbito venezolano, la revisión de las bases técnicas revela una
dependencia histórica de tablas de mortalidad estáticas que no capturan
la dinámica actual (Masjuán, 1965). Investigaciones locales han
comenzado a señalar brechas en la solvencia del sistema de previsión
social debido a esta desactualización (Moreno, 2018). Estudios recientes
de mortalidad infantil (Moreno, 2017) y análisis regionales en el
occidente (Briceño, 2024) y la región central (Godoy, 2024) subrayan la
necesidad de integrar la dimensión geográfica para una tarificación
precisa. Finalmente, la integración de modelos de proyección como el de
Lee y Carter (1992) con estructuras espaciales permite una visión
prospectiva esencial para la gestión de riesgos a largo plazo.

Este trabajo expande la frontera hacia el nivel municipal para capturar
la heterogeneidad que autores como Kennya Briceño (2024) y Daniela Godoy
(2024) han comenzado a documentar en sus análisis regionales.

La integración de herramientas como INLA (Integrated Nested Laplace
Approximation) permite realizar inferencia bayesiana de manera
computacionalmente eficiente, superando las limitaciones de tiempo de
los métodos MCMC tradicionales.

Este enfoque es vital para procesar los 335 municipios venezolanos a lo
largo de 21 años, permitiendo una visión espacio-temporal que cuantifica
no solo dónde mueren más personas, sino cómo ese riesgo evoluciona en el
tiempo.

\section{Métodos}\label{muxe9todos}

La metodología de esta investigación se estructura en un flujo de
trabajo reproducible que transita desde la curaduría de datos
administrativos hasta la inferencia bayesiana avanzada. El enfoque es
multidimensional e integra las siguientes fases técnicas:

\textbf{Curaduría de Datos:} Se implementan funciones de limpieza y
normalización de nombres geográficos para garantizar la
interoperabilidad entre los registros del Ministerio del Poder Popular
para la Salud (MPPS), las proyecciones del Instituto Nacional de
Estadística (INE) y la cartografía GADM. Esto permite un cruce preciso
de los 335 municipios a lo largo del panel 1996-2016.

\textbf{Exploración de Perfiles (K-means):} Se aplica el algoritmo de
aprendizaje no supervisado K-means sobre las trayectorias longitudinales
de las tasas de mortalidad. El objetivo es segmentar los municipios en
clusters de riesgo que compartan comportamientos dinámicos similares,
independientemente de su ubicación física, facilitando la identificación
de grupos para tarificación actuarial.

\textbf{Análisis de Autocorrelación Espacial (Moran's I):} Teóricamente,
el Índice de Moran mide el grado en que unidades geográficas cercanas
presentan valores similares (autocorrelación positiva) o disímiles
(autocorrelación negativa). Se define como una medida global de
asociación lineal entre un valor observado en una ubicación y el
promedio de los valores en su vecindad. Un valor estadísticamente
significativo permite rechazar la hipótesis nula de aleatoriedad
espacial (Spatial Randomness), confirmando que la mortalidad sigue una
estructura geográfica que debe ser modelada explícitamente.

\textbf{Modelado Bayesiano Jerárquico (BYM):} El modelo
Besag-York-Mollié es un método de estimación para áreas pequeñas que
descompone el logaritmo del riesgo relativo en dos componentes
aleatorios: un efecto espacial estructurado (\(u_i\)), que asume un
proceso autorregresivo condicional intrínseco (ICAR) basado en la
vecindad, y un efecto no estructurado (\(v_i\)), que captura la
heterogeneidad local independiente (ruido blanco). Este enfoque permite
``suavizar'' las tasas municipales, encogiendo las estimaciones
inestables de áreas poco pobladas hacia el promedio de sus vecinos, lo
que resulta en un mapa de riesgo ``puro'' mucho más confiable para el
cálculo de reservas actuariales.

\textbf{Interpolación y Visualización (IDW):} La Ponderación de
Distancia Inversa (Inverse Distance Weighting) es un método
determinístico de interpolación espacial basado en el principio de que
los valores desconocidos en un punto pueden estimarse mediante un
promedio ponderado de los valores observados en puntos cercanos. La
ponderación es inversamente proporcional a la distancia elevada a una
potencia (\(p\)), permitiendo generar superficies de riesgo continuo que
revelan núcleos de mortalidad que atraviesan las fronteras
administrativas legales, facilitando la toma de decisiones
territorializadas.

\begin{longtable}[]{@{}
  >{\raggedright\arraybackslash}p{(\linewidth - 4\tabcolsep) * \real{0.3333}}
  >{\raggedright\arraybackslash}p{(\linewidth - 4\tabcolsep) * \real{0.3333}}
  >{\raggedright\arraybackslash}p{(\linewidth - 4\tabcolsep) * \real{0.3333}}@{}}
\toprule\noalign{}
\endhead
\bottomrule\noalign{}
\endlastfoot
& & \\
Modelo / Técnica & Aplicación en este Estudio & Ventaja Actuarial \\
SMR Tradicional & Diagnóstico inicial de exceso de mortalidad &
Comparabilidad directa con estándares internacionales. \\
K-means Clustering & Segmentación de trayectorias históricas &
Identificación de grupos de riesgo para tarificación grupal. \\
Índice de Moran & Test de autocorrelación espacial & Validación de la
estructura geográfica del riesgo. \\
Modelo BYM (INLA) & Suavizamiento de riesgos relativos & Estabilidad en
la estimación para municipios de baja población. \\
Interpolación IDW & Generación de superficies de riesgo & Visualización
continua para políticas de salud y seguros. \\
\end{longtable}

\subsection{\texorpdfstring{\textbf{Diseño del Estudio y Fuentes de
Datos}}{Diseño del Estudio y Fuentes de Datos}}\label{diseuxf1o-del-estudio-y-fuentes-de-datos}

Este trabajo emplea un diseño ecológico longitudinal que analiza la
totalidad de los municipios de Venezuela durante el período 1996-2016.
La estrategia empírica se basa en la integración de tres fuentes
principales: los registros administrativos de defunciones del Ministerio
del Poder Popular para la Salud (MPPS), los datos demográficos del
Instituto Nacional de Estadística (INE) y la base cartográfica GADM
(Global Administrative Areas). ~

El desafío metodológico principal radica en la estimación de los
denominadores poblacionales. Ante la ausencia de censos anuales después
de 2011, se aplicó una tasa de crecimiento intercensal del 1.6\% anual
para proyectar las poblaciones municipales. Esta técnica, aunque
simplificada, permite mantener la comparabilidad temporal necesaria para
el análisis de tendencias de larga duración. ~

\subsubsection{\texorpdfstring{\textbf{Procedimiento de Codificación y
Procesamiento}}{Procedimiento de Codificación y Procesamiento}}\label{procedimiento-de-codificaciuxf3n-y-procesamiento}

El código implementado en R sigue un flujo de trabajo reproducible que
garantiza la transparencia exigida por la RVEA. Se inicia con la
limpieza de los registros de mortalidad, donde se eliminan
inconsistencias y se normalizan las entidades federales, asegurando que
la transición histórica (por ejemplo, de Distrito Federal a Vargas/La
Guaira) sea capturada correctamente.

\subsubsection{\texorpdfstring{\textbf{Estrategia de Modelado
Matemático}}{Estrategia de Modelado Matemático}}\label{estrategia-de-modelado-matemuxe1tico}

Exploración: Clusterización K-means para identificar perfiles de riesgo
históricos.

Autocorrelación: Índice de Moran utilizando una matriz de vecinos más
cercanos (KNN=2) para integrar territorios insulares.

Modelado: Regresión de Poisson para tendencias temporales y Modelo BYM
(Besag-York-Mollié) para riesgo relativo suavizado.

Interpolación: Método IDW para generar una superficie continua de
riesgo.

El modelo espacial bayesiano implementado es de tipo jerárquico. En el
primer nivel, se asume que el número de muertes observadas \(y_i\) sigue
una distribución de Poisson con media \(\lambda_i = E_i \theta_i\). En
el segundo nivel, se modela el logaritmo del riesgo relativo
\(\theta_i\) como :

\$\$\textbackslash eta\_i = \textbackslash log(\textbackslash theta\_i)
= \textbackslash alpha + u\_i + v\_i\$\$

Donde:

\begin{itemize}
\item
  \(\alpha\): El intercepto que representa el riesgo promedio nacional.
\item
  \(u_i\): El efecto espacial estructurado, modelado mediante un Prior
  Intrínseco Autorregresivo Condicional (iCAR), que suaviza el riesgo
  basado en la vecindad.
\item
  \(v_i\): El efecto no estructurado, que captura la sobredispersión o
  variación local no espacial.
\end{itemize}

Para la integración de territorios insulares (Margarita, Coche, Los
Roques), se utilizó una matriz de vecindad basada en los 2 vecinos más
cercanos (KNN=2), evitando así el aislamiento de estos nodos en el grafo
de INLA.

\section{Resultados}\label{resultados}

Los resultados revelan una geografía de la mortalidad heterogénea y
altamente estructurada. A continuación, se presenta la evidencia
partiendo de la identificación de grupos de riesgo hasta la
visualización de la superficie continua de mortalidad nacional. Los
hallazgos confirman que el riesgo de muerte en Venezuela no se
distribuye de manera aleatoria, sino que obedece a patrones espaciales
persistentes.

\subsection{Análisis Descriptivo de la Evolución
Temporal}\label{anuxe1lisis-descriptivo-de-la-evoluciuxf3n-temporal}

Antes de la segmentación algorítmica, se observa el comportamiento de
las tasas de mortalidad resumidas por los 24 estados del país frente a
la tendencia media nacional, el análisis de distribución anual permite
identificar la presencia de valores atípicos y el ensanchamiento de las
brechas territoriales.

\begin{figure}

\centering{

\pandocbounded{\includegraphics[keepaspectratio]{Art_Angel_Plantilla_RVEA_files/figure-pdf/fig-boxplot-anual-1.pdf}}

}

\caption{\label{fig-boxplot-anual}Distribución de las tasas de
mortalidad municipal por año (1996-2016). Los puntos representan
municipios individuales; las cajas muestran los cuartiles y la mediana
nacional.}

\end{figure}%

La Figura anterior revela un incremento progresivo en la dispersión de
las tasas de mortalidad a partir de mediados de la década de 2000. Los
municipios atípicos (outliers en rojo) sugieren crisis locales que las
métricas nacionales no logran capturar.

Para evitar la saturación de las leyendas, el análisis se divide en dos
bloques regionales organizados alfabéticamente.

\begin{figure}

\centering{

\pandocbounded{\includegraphics[keepaspectratio]{Art_Angel_Plantilla_RVEA_files/figure-pdf/fig-tendencia-inicial-parte1-1.pdf}}

}

\caption{\label{fig-tendencia-inicial-parte1}Tendencias de mortalidad
promedio por Estado (Parte 1: Grupos A-F). La línea negra punteada
resalta el promedio nacional.}

\end{figure}%

\begin{figure}

\centering{

\pandocbounded{\includegraphics[keepaspectratio]{Art_Angel_Plantilla_RVEA_files/figure-pdf/fig-tendencia-inicial-parte2-1.pdf}}

}

\caption{\label{fig-tendencia-inicial-parte2}Tendencias de mortalidad
promedio por Estado (Parte 2: Grupos G-Z). La línea negra punteada
resalta el promedio nacional.}

\end{figure}%

\subsection{Segmentación por Clusters de
Riesgo}\label{segmentaciuxf3n-por-clusters-de-riesgo}

La aplicación del algoritmo K-means sobre las trayectorias temporales de
las tasas de mortalidad permitió identificar cuatro perfiles
diferenciados en el territorio venezolano. Esta segmentación es
fundamental para la tarificación actuarial, ya que permite agrupar
municipios no solo por su ubicación, sino por su comportamiento dinámico
ante el riesgo de muerte.

\begin{figure}

\centering{

\includegraphics[width=1\linewidth,height=\textheight,keepaspectratio]{Art_Angel_Plantilla_RVEA_files/figure-pdf/fig-clusters-1.pdf}

}

\caption{\label{fig-clusters}Mapa de clusters de riesgo de mortalidad
municipal (1996-2016).}

\end{figure}%

La interpretación de estos clusters revela una división geográfica
clara: el Cluster 1 tiende a concentrar municipios urbanos con
infraestructuras de salud más densas pero también con mayores riesgos de
causas externas, mientras que los clusters rurales muestran trayectorias
de mortalidad asociadas a enfermedades endémicas o falta de acceso a
servicios básicos.

\subsection{Análisis Temporal por Cluster
Identificado}\label{anuxe1lisis-temporal-por-cluster-identificado}

Al separar las trayectorias por cluster, se evidencia la capacidad del
algoritmo para agrupar municipios con dinámicas de salud pública
divergentes. Cada panel muestra los municipios de un cluster particular
coloreados por su estado de origen.

\begin{figure}

\centering{

\pandocbounded{\includegraphics[keepaspectratio]{Art_Angel_Plantilla_RVEA_files/figure-pdf/fig-evolucion-clusters-separados-1.pdf}}

}

\caption{\label{fig-evolucion-clusters-separados}Trayectorias de
mortalidad desagregadas por Cluster. Las líneas tenues representan
municipios coloreados por estado; la línea negra gruesa indica la
tendencia promedio del cluster.}

\end{figure}%

\subsection{\texorpdfstring{\textbf{Análisis de Autocorrelación Espacial
(Moran's
I)}}{Análisis de Autocorrelación Espacial (Moran's I)}}\label{anuxe1lisis-de-autocorrelaciuxf3n-espacial-morans-i}

Para validar la necesidad de un modelo espacial bayesiano, se aplicó el
Índice de Moran sobre las tasas de mortalidad de 2016. Un valor positivo
y significativo confirma que la mortalidad en Venezuela se distribuye en
``hotspots'' o conglomerados geográficos. ~

El Índice de Moran obtenido (\(0.094\)) con un p-valor significativo
(\(0.024\)) rechaza la hipótesis de aleatoriedad espacial. Esto implica
que factores locales (clima, economía regional, políticas de salud
estatales) influyen en la mortalidad de municipios adyacentes de manera
coordinada.

\subsubsection{Evolución de la Razón de Mortalidad Estandarizada
(SMR)}\label{evoluciuxf3n-de-la-razuxf3n-de-mortalidad-estandarizada-smr}

El cálculo del SMR permite identificar municipios con excesos de
mortalidad persistentes. Un SMR de 2.0 indica que el municipio tiene el
doble de muertes de lo esperado dada su estructura poblacional y la tasa
nacional de referencia.

\begin{figure}[H]

{\centering \includegraphics[width=1\linewidth,height=\textheight,keepaspectratio]{Art_Angel_Plantilla_RVEA_files/figure-pdf/visualizacion-smr-1.pdf}

}

\caption{Evolución espacio-temporal del SMR (1996-2016).}

\end{figure}%

\begin{verbatim}
# A tibble: 15 x 6
   EST            Muni_Clean Años_Alto_Riesgo
   <chr>          <chr>                 <int>
 1 MIRANDA        CHACAO                   20
 2 YARACUY        SAN FELIPE               20
 3 TRUJILLO       VALERA                   20
 4 TACHIRA        SAN CRIST~               20
 5 NUEVA ESPARTA  MARINO                   20
 6 ARAGUA         GIRARDOT                 20
 7 MERIDA         LIBERTADOR               20
 8 CARABOBO       NAGUANAGUA               20
 9 TRUJILLO       TRUJILLO                 20
10 DISTRITO CAPI~ LIBERTADOR               20
11 PORTUGUESA     ARAURE                   20
12 GUARICO        JUAN GERM~               18
13 MIRANDA        BARUTA                   17
14 SUCRE          BERMUDEZ                 17
15 ZULIA          MARACAIBO                20
# i 3 more variables: Total_Años <int>,
#   SMR_Promedio <dbl>,
#   Persistencia_Pct <dbl>
\end{verbatim}

La persistencia del alto riesgo en municipios urbanos como Chacao y San
Cristóbal sugiere que estas áreas actúan como receptores de pacientes de
alta complejidad o presentan perfiles de envejecimiento mucho más
avanzados que el promedio nacional, lo que debe ser considerado en la
tarificación de seguros de salud y vida locales. ~

\subsubsection{\texorpdfstring{\textbf{Regresión de Poisson y Tendencias
de
Aceleración}}{Regresión de Poisson y Tendencias de Aceleración}}\label{regresiuxf3n-de-poisson-y-tendencias-de-aceleraciuxf3n}

Para cuantificar el cambio temporal, se ajustó un modelo de regresión de
Poisson que controla por el efecto de los clusters y el crecimiento
poblacional (offset).

\begin{verbatim}
# A tibble: 4 x 4
  Termino  Riesgo_Relativo  P_Valor
  <chr>              <dbl>    <dbl>
1 Tiempo             1.01  0       
2 Cluster2           0.703 0       
3 Cluster3           1.03  6.82e-83
4 Cluster4           0.743 0       
# i 1 more variable: Interpretacion <chr>
\end{verbatim}

\begin{center}
\includegraphics[width=1\linewidth,height=\textheight,keepaspectratio]{Art_Angel_Plantilla_RVEA_files/figure-pdf/regresion-poisson-1.pdf}
\end{center}

\begin{verbatim}
# A tibble: 5 x 3
  EST     Muni_Clean         Cambio_Anual_Pct
  <chr>   <chr>                         <dbl>
1 ARAGUA  OCUMARE LA COSTA ~             52.1
2 BARINAS ANDRES ELOY BLANCO             48.0
3 ARAGUA  FRANCISCO LINARES              24.3
4 TACHIRA SAN JUDAS TADEO                22.2
5 BOLIVAR PADRE PEDRO CHIEN              16.0
\end{verbatim}

El modelo revela una tendencia nacional de incremento del riesgo del
1.4\% anual durante el período estudiado. Sin embargo, esta tendencia no
es uniforme. Municipios específicos como Ocumare la Costa de Oro y
Andrés Eloy Blanco muestran tasas de aceleración críticas que superan el
40\% anual en ciertos períodos, lo cual podría estar vinculado a eventos
de violencia local o crisis de servicios de salud regionales. ~

\subsubsection{\texorpdfstring{\textbf{Modelado Bayesiano de Riesgo
Suavizado
(INLA)}}{Modelado Bayesiano de Riesgo Suavizado (INLA)}}\label{modelado-bayesiano-de-riesgo-suavizado-inla}

El modelo Besag-York-Mollié (BYM) es la pieza central de este estudio.
Al ``limpiar'' el ruido aleatorio, permite ver la estructura de riesgo
latente de Venezuela.

\begin{figure}[H]

{\centering \includegraphics[width=1\linewidth,height=\textheight,keepaspectratio]{Art_Angel_Plantilla_RVEA_files/figure-pdf/modelado-inla-1.pdf}

}

\caption{Mapa de Riesgo Relativo Suavizado (Modelo Bayesiano).}

\end{figure}%

El mapa de Riesgo Relativo Suavizado (RR) muestra un arco de alta
mortalidad que recorre el centro-norte del país y se extiende hacia
ciertas zonas fronterizas. Este riesgo estructural es independiente de
las fluctuaciones temporales anuales y representa la base de riesgo que
las aseguradoras y fondos de pensiones deben considerar para el cálculo
de sus reservas técnicas territoriales. ~ ~\textbf{Análisis del Gráfico
de Oruga} Para una interpretación técnica rigurosa de los resultados del
modelo Bayesiano, no basta con observar el valor medio del riesgo
relativo por municipio en una escala cromática. Es imperativo evaluar la
incertidumbre estadística asociada a cada estimación, especialmente en
unidades territoriales con poblaciones reducidas donde la volatilidad de
los datos administrativos suele ser mayor. El Gráfico de Oruga (Figura
Figure~\ref{fig-caterpillar}) permite jerarquizar los municipios según
su nivel de riesgo estructural, facilitando la identificación de
aquellos donde el exceso o defecto de mortalidad es estadísticamente
significativo.

\begin{figure}

\centering{

\pandocbounded{\includegraphics[keepaspectratio]{Art_Angel_Plantilla_RVEA_files/figure-pdf/fig-caterpillar-1.pdf}}

}

\caption{\label{fig-caterpillar}Gráfico de Oruga (Caterpillar Plot) de
los Riesgos Relativos Municipales. Se muestran las medias posteriores y
los intervalos de credibilidad del 95\%.}

\end{figure}%

La inspección de la Figura Figure~\ref{fig-caterpillar} revela una
separación clara entre los municipios con riesgos estructurales críticos
y aquellos con perfiles protectores. Los intervalos de credibilidad que
no intersectan la línea de unidad (\(RR=1\)) representan áreas de
prioridad actuarial. Este análisis de significancia complementa la
visión geográfica, permitiendo que la tarificación territorial y la
constitución de reservas no se basen en picos de ruido aleatorio, sino
en desviaciones sistemáticas y estadísticamente robustas de la
mortalidad nacional. Los municipios resaltados en rojo constituyen el
núcleo del exceso de mortalidad que debe ser abordado mediante ajustes
técnicos específicos en las tablas actuariales subnacionales.

~\textbf{Mapa de Probabilidad de Excedencia~}

Una limitación de los mapas de Riesgo Relativo (RR) es que no indican
cuán seguros estamos de que el riesgo sea realmente alto. El Mapa de
Probabilidad de Excedencia soluciona esto calculando
\(P(RR > 1 | \text{datos})\).

Si la probabilidad es cercana a 1, tenemos una certeza casi absoluta de
que el municipio presenta un exceso de mortalidad estructural.

\begin{figure}

\centering{

\pandocbounded{\includegraphics[keepaspectratio]{Art_Angel_Plantilla_RVEA_files/figure-pdf/fig-probabilidad-mapa-1.pdf}}

}

\caption{\label{fig-probabilidad-mapa}Mapa de Probabilidad de Excedencia
P(RR \textgreater{} 1). Las zonas en rojo oscuro indican una certeza
estadística \textgreater95\% de exceso de mortalidad.}

\end{figure}%

\subsubsection{\texorpdfstring{\textbf{Mapa de Calor Continuo
(Interpolación
IDW)}}{Mapa de Calor Continuo (Interpolación IDW)}}\label{mapa-de-calor-continuo-interpolaciuxf3n-idw}

La interpolación por ponderación de distancia inversa (IDW) permite
visualizar la transición del riesgo como una superficie sólida, lo que
facilita la identificación de fronteras epidemiológicas. ~

\begin{verbatim}
[inverse distance weighted interpolation]
\end{verbatim}

\begin{figure}[H]

{\centering \includegraphics[width=1\linewidth,height=\textheight,keepaspectratio]{Art_Angel_Plantilla_RVEA_files/figure-pdf/heatmap-interpolado-1.pdf}

}

\caption{Mapa de Calor Continuo de Mortalidad (Interpolación IDW).}

\end{figure}%

\subsection{Proyección de Mortalidad}\label{proyecciuxf3n-de-mortalidad}

La capacidad predictiva es el pilar de la solvencia actuarial.
Utilizando la estructura espacio-temporal aprendida por el modelo
Bayesiano, se han proyectado las tasas de mortalidad para el quinquenio
posterior al estudio. Esta proyección asume una ``inercia temporal''
(Random Walk de orden 1) que captura la tendencia de degradación o
mejora de las condiciones de salud, permitiendo a los actuarios
anticipar el costo futuro de las pensiones y seguros de vida.

\begin{figure}[H]

{\centering \includegraphics[width=1\linewidth,height=\textheight,keepaspectratio]{Art_Angel_Plantilla_RVEA_files/figure-pdf/Proyeccion-1.pdf}

}

\caption{Proyección de Mortalidad 2021 basada en modelo
espacio-temporal.}

\end{figure}%

\section{Discusión y Conclusiones}\label{discusiuxf3n-y-conclusiones}

\subsection{\texorpdfstring{\textbf{Discusión}}{Discusión}}\label{discusiuxf3n}

Los resultados de este análisis histórico y espacio-temporal confirman
que Venezuela posee una geografía de la mortalidad altamente heterogénea
y dinámica. La identificación de un incremento nacional del riesgo del
1.4\% anual, ajustado por estructura espacial, es un hallazgo de suma
gravedad para la solvencia de largo plazo del sistema de pensiones. Si
las tablas de mortalidad no se actualizan anualmente incorporando este
factor de aceleración, el pasivo actuarial del país continuará creciendo
de manera oculta e insostenible. ~

La persistencia del alto riesgo en los municipios del arco central
(Distrito Capital, Miranda, Aragua, Carabobo) sugiere que la
urbanización en Venezuela no ha funcionado como un factor protector,
como indica la teoría de la transición epidemiológica clásica, sino que
ha concentrado riesgos asociados a la violencia, accidentes y
enfermedades crónicas no transmisibles. Municipios como Chacao y
Libertador (DC), a pesar de contar con la mayor densidad de centros de
salud, presentan los SMR más altos del país, lo que podría explicarse
por un fenómeno de selección: personas con enfermedades graves de todo
el país migran a la capital para recibir tratamiento, inflando las
estadísticas locales de defunción, o bien por un perfil de
envejecimiento in situ mucho más acelerado. ~

Desde la perspectiva del modelado, el uso de INLA y el modelo BYM ha
demostrado ser superior a la inspección de tasas crudas. El mapa
suavizado (RR) elimina los picos artificiales en municipios pequeños y
revela que el riesgo tiene una estructura espacial coherente. Esto
valida las investigaciones de Arlet Moreno (2017) sobre mortalidad
infantil, quien también detectó heterogeneidad por entidad federal, pero
ahora expandida a una resolución 335 veces mayor.

\subsection{\texorpdfstring{\textbf{Conclusiones y
Recomendaciones}}{Conclusiones y Recomendaciones}}\label{conclusiones-y-recomendaciones}

El análisis exhaustivo de la mortalidad en Venezuela (1996-2016) permite
concluir que el país enfrenta una crisis demográfica de carácter
territorial. La obsolescencia de las bases técnicas, heredada de la
década de 1960, ha sido superada por una realidad donde el riesgo de
muerte aumenta anualmente y se concentra geográficamente en polos de
alta vulnerabilidad. Como miembro del Comité Editorial de la RVEA y
tutor de investigaciones fundamentales en este volumen, presento las
siguientes recomendaciones estratégicas para el sector actuarial y los
decisores de política pública: ~

\begin{enumerate}
\def\labelenumi{\arabic{enumi}.}
\item
  \textbf{Abandono de Tablas Estáticas:} Se recomienda al IVSS y a la
  Superintendencia de la Actividad Aseguradora (Sudeaseg) la transición
  obligatoria hacia tablas dinámicas que incorporen factores de mejora
  (o deterioro) de la mortalidad, como el modelo Lee-Carter o el modelo
  BYM aquí presentado. ~
\item
  \textbf{Territorialización de Reservas:} Los fondos de pensiones y las
  compañías de seguros deben evaluar la posibilidad de aplicar recargos
  o descuentos zonales basados en el mapa de Riesgo Relativo Suavizado.
  Un asegurado en Chacao no representa el mismo riesgo financiero que
  uno en un municipio rural del estado Los Andes. ~
\item
  \textbf{Monitoreo de la Aceleración:} El incremento del 1.4\% anual
  detectado es un ``cisne negro'' demográfico que erosiona la solvencia.
  Se debe establecer un Observatorio de Mortalidad en el CVEA que
  procese anualmente los registros administrativos para detectar cambios
  bruscos en las tendencias municipales. ~
\item
  \textbf{Integración de Datos y Código Abierto:} Siguiendo la política
  de la RVEA, se insta a los investigadores a utilizar el marco de
  Quarto y R para garantizar que sus modelos sean replicables y
  auditables por terceros, fortaleciendo la confianza en los
  diagnósticos actuariales del país. ~
\end{enumerate}

Este trabajo no solo cierra una brecha de conocimiento de dos décadas,
sino que proporciona el mapa de ruta técnico para la construcción de un
sistema de previsión social más resiliente, basado en datos reales y
metodologías de estándar internacional. ~

\section*{Agradecimientos}\label{agradecimientos}
\addcontentsline{toc}{section}{Agradecimientos}

El autor desea expresar su profundo agradecimiento a la Escuela de
Estadística y Ciencias Actuariales de la UCV y al Centro Venezolano de
Estudios Actuariales (CVEA) por proporcionar el entorno académico
propicio para el desarrollo de esta investigación. Asimismo, se reconoce
el valioso aporte de los tesistas Daylin Moreno, José Raúl Gálvez,
Daniel Azuaje y Arlet Moreno, cuyos trabajos de grado han sido pilares
fundamentales para el diagnóstico sistémico presentado en este volumen
inaugural de la RVEA.

\section*{Referencias}\label{referencias}
\addcontentsline{toc}{section}{Referencias}

\begin{itemize}
\item
  Besag, J., York, J., \& Mollié, A. (1991). Bayesian image restoration,
  with two applications in spatial statistics. Annals of the Institute
  of Statistical Mathematics, 43(1), 1-20.
\item
  Blangiardo, M., \& Cameletti, M. (2015). Spatial and Spatio-temporal
  Bayesian Models with R-INLA. John Wiley \& Sons.
\item
  Briceño, K. (2024). Análisis regional de la mortalidad en el occidente
  venezolano. Revista Venezolana de Actuariado, 1(1), 45-62.
\item
  Clayton, D., \& Kaldor, J. (1987). Empirical Bayes estimates for
  cancer mortality rates at the local level. Biometrics, 43(3), 671-681.
\item
  Godoy, D. (2024). Heterogeneidad de la mortalidad en la región
  central: Un enfoque municipal. Revista Venezolana de Actuariado, 1(1),
  63-78.
\item
  Lawson, A. B. (2018). Bayesian Disease Mapping: Hierarchical Modeling
  in Spatial Epidemiology. CRC Press.
\item
  Lee, R. D., \& Carter, L. R. (1992). Modeling and Forecasting U.S.
  Mortality. Journal of the American Statistical Association, 87(419),
  659-671.
\item
  Masjuán, V. (1965). Tablas de mortalidad para la población venezolana.
  Caracas: Escuela de Estadística y Ciencias Actuariales, UCV.
\item
  Moreno, A. (2017). Modelado espacial de la mortalidad infantil en
  Venezuela mediante métodos Bayesianos. Trabajo de Grado, Universidad
  Central de Venezuela.
\item
  Moreno, D. (2018). Evaluación de la solvencia del sistema de pensiones
  en Venezuela: Un análisis de longevidad. Trabajo de Grado, Universidad
  Central de Venezuela.
\item
  Rue, H., Martino, S., \& Chopin, N. (2009). Approximate Bayesian
  inference for latent Gaussian models by using integrated nested
  Laplace approximations. Journal of the Royal Statistical Society:
  Series B (Statistical Methodology), 71(2), 319-392.
\item
  Tobler, W. R. (1970). A Computer Movie Simulating Urban Growth in the
  Detroit Region. Economic Geography, 46, 234-240.
\item
  Waller, L. A., \& Gotway, C. A. (2004). Applied Spatial Statistics for
  Public Health Data. John Wiley \& Sons.
\end{itemize}

\begin{center}\rule{0.5\linewidth}{0.5pt}\end{center}




\end{document}
